\chapter {Aims and requirements}
Based on the previous analysis of innovative ideas this chapter will define the functional and non-functional requirements on the developed solution. As well as the innovative features, the solution has to offer also the traditional functions of bank information systems.

\section{Traditional requirements on online banking}
Traditional online banking application has the following set of functions:

\begin{enumerate}
	\item Account management
	\item Transaction management
	\item Data exports to PDF, CSV or Excel formats.
\end{enumerate}

\section{Requirements on innovative features}
Based on the analysis in the previous chapter, the following set of features and functions should be implemented in the solution.

\begin{itemize}
	\item  Personal Finance Management
	\begin{itemize}
		\item Allow the clients to categorize transactions manually.
		\item Allow the clients to use set of standard tags (or categories) as well as define own tags.
		\item Provide an automatic classification engine.
		\item Allow the clients to visualize the data from PFM, such as category repartition certain time period.
		\item Provide the payment management functionality for the client implementing a calendar of payments.
	\end{itemize}
	\item  Electronic vault
	Provide a electronic vault where the documents for client and his adviser will be stored. This vault will provide the following functions:
	\begin{itemize}
		\item  Only authenticated client will be able to upload any files to the vault
		\item  Only authenticated client will be able to download any files from the vault
		\item  Generated documents such as accounts overview will be stored directly in the vault.
	\end{itemize}
	\item Open Data
	Clients will have the possibility to delegate the rights to access certain data to third party applications.
	This access will be revokable and clients will have the possibility to revoke the access any time. OAuth protocol will be used as the industry standard for rights delegation.
	\item  Using face recognition in authentication process
	Clients will be able to use face recognition to simplify the process of authentication.
\end{itemize}

\section{Technical requirements}

\subsection{Required technologies}
OCTO Technology as the ordering party of this implementation demands the use of .NET technology stack. Specifically the solution should use the following technologies:
\begin{itemize}
	\item  Microsoft Silverlight as the web interface.
	\item  Windows Phone 7 as the mobile platform.
	\item  The back-end platform of the solution has to be deploy-able to Internet Information Server as well as Windows Azure platform.
\end{itemize}

\subsection{Decoupled architecture}
Provided implementation has to be designed taking into account modularity and decoupled architecture. The modules which build the solution have to replaceable by other modules with different implementations.

\subsection{Use of Cloud technologies}
Cloud platforms have several advantages such as lowering the costs for maintenance and easy scalability.
The solution should be deploy-able to Cloud based infrastructure. Because the solution is designed for .NET platform, Windows Azure is the platform which should be targeted.  

\subsection{Testable architecture}
OCTO Technology demands to place considerable effort on the unit testing of back-end and front-end of the application. Besides traditional unit testing approach Parametrized unit testing should be performed. One of the aims of this effort is to try the use of PEX\ref{tech:parametrized_testing} platform and evaluate it's usefulness for business layer testing.